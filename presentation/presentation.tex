\documentclass{beamer}

\usepackage{hyperref}
\usepackage{listings}
\usepackage{color}

% Colours for beamer.
\setbeamercolor{frametitle}{fg=orange}
\setbeamertemplate{itemize item}{\color{orange}$\blacksquare$}
\setbeamertemplate{itemize subitem}{\color{orange}$\blacktriangleright$}

% Colours for syntax highlighting
\definecolor{syntax_red}{rgb}{0.7, 0.0, 0.0} % For strings
\definecolor{syntax_green}{rgb}{0.15, 0.5, 0.25} % For comments
\definecolor{syntax_purple}{rgb}{0.6, 0.0, 0.45} % For keywords


% Haskell settings for lstlisting
\lstset{language=Haskell,
basicstyle=\ttfamily,
keywordstyle=\color{syntax_purple}\bfseries,
stringstyle=\color{syntax_red},
commentstyle=\color{syntax_green},
numbers=none,
numberstyle=\color{black},
stepnumber=1,
numbersep=10pt,
tabsize=4,
showspaces=false,
showstringspaces=false}

\author{
  Beck, Calvin\\
  \href{mailto:hobbes@ualberta.ca}{hobbes@ualberta.ca}
}

\begin{document}

\begin{frame}
  \frametitle{Formal Verification and Coq}
  \maketitle
\end{frame}

\section{Introduction}

\begin{frame}
  \frametitle{What is this Talk about?}

  Formal verification, mostly! This presentation hopes to address the following:

  \begin{itemize}
  \item Why should you care / be interested?
  \item Briefly cover some of the methods.
  \item Make proof assistants more accessible.
  \item Give some rough intuitions about how these systems work.
  \end{itemize}
\end{frame}

\section{Preface}

\begin{frame}[fragile]
  \frametitle{Type Signatures}

  \texttt{x} has type \texttt{A}:
  \begin{lstlisting}[frame=single, language=Haskell, breaklines=true]
    x :: A
  \end{lstlisting}

  \pause

  \texttt{x} is an integer, \texttt{y} is a list of integers.
  \begin{lstlisting}[frame=single, language=Haskell, breaklines=true]
    x :: Integer
    y :: [Integer]
  \end{lstlisting}

\end{frame}

\begin{frame}[fragile]
  \frametitle{Type Signatures Continued...}

  Functions have types too!

  \begin{lstlisting}[frame=single, language=Haskell, breaklines=true]
    (+) :: Integer -> Integer -> Integer
  \end{lstlisting}

\end{frame}

\begin{frame}[fragile]
  \frametitle{Type Signatures Continued...}

  Types can also be polymorphic. The identity function, \texttt{id}, may take any type as an argument.

  \begin{lstlisting}[frame=single, language=Haskell, breaklines=true]
    id :: a -> a
    id x = x
  \end{lstlisting}
\end{frame}

\begin{frame}[fragile]
  \frametitle{Lambda Calculus}

  Lambda terms:

  \begin{itemize}
  \item Variables: ``\texttt{x}'' and such
  \item Lambda abstraction: $(\lambda x . t)$ where $t$ is another lambda term. $x$ is an argument, $t$ is the ``body''
  \item Application: $(ts)$
  \end{itemize}

  Combine as you see fit!
\end{frame}

\begin{frame}[fragile]
  \frametitle{Beta Reduction}

  Beta reduction is just substitution.

  \[\mathtt{id} = (\lambda x . x)\]

  Substituting $x$ for $t$...

  \[(\lambda x . x) t = t\]
\end{frame}

\section{Formal Verification: What it be?}

\begin{frame}
  \frametitle{Formal Verification: What it be?}

  The use of formal methods to prove that programs are correct

  \begin{itemize}
  \item<2-> Want programs to be correct
    \begin{itemize}
    \item Almost everything has a computer in it now
    \item Incorrect programs can be dangerous
    \item Bugs can be expensive
    \end{itemize}
  \item<3-> Mathematicians want computers to verify their proofs as well.
  \end{itemize}
\end{frame}

\begin{frame}
  \frametitle{Some Methods}

  \begin{itemize}
  \item<1-> Checking by hand... Manual labor :(
  \item<2-> Model checking
    \begin{itemize}
    \item Essentially checking every possible state of your program
    \item ``Proof by exhaustion''
    \item This can be computationally expensive
    \item Works best on small F.S.M.s.
    \end{itemize}
  \item<3-> Type Checking
    \begin{itemize}
    \item Types provide guarantees about how values behave
    \item Most languages do this badly (Java, Python)
    \item Some are good, but still limited (Haskell)
    \end{itemize}
  \item<4-> Theorem Proving
    \begin{itemize}
    \item Mathematical proofs for great justice
    \item Use the computer to check the proofs
    \item This actually boils down to extended type checking
    \end{itemize}
  \end{itemize}
\end{frame}

\begin{frame}
  \frametitle{Levels of Abstraction}

  \begin{itemize}
  \item<1-> High Level: Algorithms? Correct implementation?
  \item<2-> Low Level: Check machine code?
  \item<3-> Hardware?
  \item<4-> Mix and match!
  \end{itemize}

  \onslide<5->

  \alert{We'll focus on high level stuff!}
\end{frame}

\section{Coq}

\begin{frame}
  \frametitle{The Coq Proof Assistant}

  We're going to be looking at one of the staples of the industry, called Coq. So named because of:

  \begin{itemize}
  \item<1-> Tradition of naming programming languages after animals (OCaml). Xavier Leroy created both OCaml and Coq
  \item<2-> French. Xavier Leroy is from France
  \item<3-> CoC: Calculus of Constructions
  \item<4-> Thierry Coquand is the creator of CoC
  \item<5-> Basically the universe is trying to make this talk awkward
  \end{itemize}
\end{frame}

\section{Examples}

\begin{frame}
  \frametitle{Examples}

  \alert{\huge{MOVING ON TO EXAMPLES!}}
\end{frame}

\section{Theory}

\begin{frame}
  \frametitle{Theory 'n Stuff}

 \alert{\huge{Coq is basically just a type-checker!}}

\end{frame}

\begin{frame}
  \frametitle{Theory 'n Stuff}

  \begin{itemize}
  \item<1-> What does type checking have to do with proving theorems?
    \onslide<2->

    \begin{itemize}
    \item \alert{EVERYTHING due to the Curry-Howard isomorphism!}
    \end{itemize}

  \end{itemize}

\end{frame}

\begin{frame}[fragile]
  \frametitle{Theory 'n Stuff}
  \begin{itemize}
  \item<1-> Curry-Howard isomorphism relates programs to proofs.
    \begin{itemize}
    \item Specifically it relates terms of the simply-typed lambda calculus to intuitionistic logic.
    \item Coq actually uses the ``Calculus of Constructions''. It's another lambda calculus, but has some special sauce which enable quantifiers and has some other nice properties.
    \item Simply-typed lambda calculus still provides some good intuition, however.
    \end{itemize}

  \item<2-> Types are propositions. For instance, the type:

\begin{lstlisting}[frame=single, language=Haskell, breaklines=true]
      a -> b
\end{lstlisting}

    corresponds to the proposition $a \rightarrow b$. ``$a$ implies $b$''.

  \item<3-> A program inhabiting that type is an existence proof of the proposition.

    \begin{itemize}
    \item Roughly speaking the program implements the proposition, so it demonstrates that the proposition is true.
    \end{itemize}
  \end{itemize}

\end{frame}

\begin{frame}[fragile]
  \frametitle{Curry-Howard: A Brief Introduction}

  \begin{itemize}
  \item<1-> Any type which is inhabited (has a value) represents a provable proposition
  \item<2-> Implication is represented with ``\texttt{->}'' in types

    \begin{itemize}
    \item If you have a value of the first type, then you can produce a value of the second type!
    \end{itemize}

  \item<3-> Conjunction ``$A \wedge B$'' corresponds to a tuple ``\texttt{(a, b)}''

    \begin{itemize}
    \item Both \texttt{a} and \texttt{b} have to be inhabited in order for \texttt{(a, b)} to be inhabited.
    \end{itemize}

  \item<4-> Disjunction ``$A \vee B$'' corresponds to \texttt{Either a b}

\begin{lstlisting}[frame=single, language=Haskell, breaklines=true]
data Either a b = Left a | Right b
\end{lstlisting}

    If either \texttt{a} or \texttt{b} has a value then \texttt{Either a b} can have a value.
  \end{itemize}
\end{frame}

\begin{frame}[fragile]
  \frametitle{Curry-Howard: Not Quite Brief Enough for one Slide}

  \begin{itemize}
  \item<1-> False is an uninhabited type. We call this type \texttt{Void}
    \begin{itemize}
    \item If the type can't have a value, then it can not be ``true''.
    \item Any false proposition is equivalent to \texttt{Void}, e.g., $a \rightarrow b$.
    \end{itemize}
  \item<2-> Negation is given by \texttt{a -> Void}

    \begin{itemize}
    \item If \texttt{a} is \texttt{Void} then it is inhabited by \texttt{id :: Void -> Void}
    \item Otherwise \texttt{a -> Void} must be uninhabited, since a function must return a value when given a value.
    \end{itemize}
  \end{itemize}
\end{frame}

\begin{frame}
  \frametitle{The Problem of Non-termination}
  \begin{itemize}
  \item If programs don't have to terminate every type is inhabited by an infinite loop!
  \begin{itemize} \item \alert{Every proposition is true, and that's not useful at all!} \end{itemize}
  \end{itemize}
\end{frame}
\end{document}
