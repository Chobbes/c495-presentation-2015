\documentclass{beamer}

\usepackage{listings}
\usepackage{color}

% Colours for syntax highlighting
\definecolor{syntax_red}{rgb}{0.7, 0.0, 0.0} % For strings
\definecolor{syntax_green}{rgb}{0.15, 0.5, 0.25} % For comments
\definecolor{syntax_purple}{rgb}{0.6, 0.0, 0.45} % For keywords


% Haskell settings for lstlisting
\lstset{language=Haskell,
basicstyle=\ttfamily,
keywordstyle=\color{syntax_purple}\bfseries,
stringstyle=\color{syntax_red},
commentstyle=\color{syntax_green},
numbers=none,
numberstyle=\color{black},
stepnumber=1,
numbersep=10pt,
tabsize=4,
showspaces=false,
showstringspaces=false}

\author{
  Beck, Calvin\\
  \email{hobbes@ualberta.ca}
}

\begin{document}

\begin{frame}
  \frametitle{Formal Verification and Coq}
  \maketitle
\end{frame}

\section{Introduction}

\begin{frame}
  \frametitle{What is this Talk about?}

  Formal verification, mostly! This presentation hopes to address the following:

  \begin{itemize}
  \item Why should you care / be interested?
  \item Briefly cover some of the methods.
  \item Make proof assistants more accessible.
  \item Give some rough intuitions about how these systems work.
  \end{itemize}
\end{frame}


\section{Preface}

\begin{frame}[fragile]
  \frametitle{Type Signatures}

    \begin{lstlisting}[frame=single, language=Haskell, breaklines=true]
      x :: A
    \end{lstlisting}

    \begin{lstlisting}[frame=single, language=Haskell, breaklines=true]
      (+) :: Integer -> Integer -> Integer
    \end{lstlisting}
\end{frame}

\begin{frame}[fragile]
  \frametitle{Type Signatures Continued...}

  \begin{lstlisting}[frame=single, language=Haskell, breaklines=true]
    id :: a -> a
  \end{lstlisting}
\end{frame}


\begin{frame}[fragile]
  \frametitle{Type Signatures Continued...}

  \begin{lstlisting}[frame=single, language=Haskell, breaklines=true]
    id :: a -> a
    id x = x
  \end{lstlisting}
\end{frame}

\end{document}
